% \iffalse meta-comment
%
% Copyright 2016
% The LaTeX3 Project and any individual authors listed elsewhere
% in this file.
%
% It may be distributed and/or modified under the conditions of
% the LaTeX Project Public License (LPPL), either version 1.3c of
% this license or (at your option) any later version.  The latest
% version of this license is in the file:
%
%   http://www.latex-project.org/lppl.txt
%
%
%<*driver>
\ProvidesFile{umath2e.dtx}[2016/02/13 v1.0 Unicode mathematics support]
\documentclass{ltxdoc}
\GetFileInfo{umath2e.dtx}
\begin{document}
\title{\filename\\(Unicode mathematics support)}
\author{Will Robertson}
\date{\filedate}
\maketitle
\setcounter{tocdepth}{2}
\tableofcontents
\DocInput{\filename}
\end{document}
%</driver>
% \fi
%
%
% \section{Overview}
%
% 
%
% \StopEventually{}
%
% \section{Implementation}
%
% \subsection{\TeX}
% 
%    \begin{macrocode}
%<*tex>
%    \end{macrocode}
%
%    \begin{macrocode}
\ProvidesPackage{umath2e}

% font loading commands

\ifdefined\directlua

\RequirePackage{luaotfload}

\directlua{
require('font-callback-mapping.lua')
}

\newcommand\umath@fonthook[1]{%
  \edef\@tmpa{#1}\def\@tmpb{}%
  \ifx\@tmpa\@tmpb\else
    \directlua{mathmapstr = '#1'}%
  \fi}

\def\extract@font{%
  \get@external@font
  \csname \curr@fontshape+preloadhook\endcsname
  \global\expandafter\font\font@name\external@font\relax
  \font@name \relax
  \csname \f@encoding+\f@family \endcsname
  \csname \curr@fontshape       \endcsname
  \umath@fonthook{nil}
  \relax
}

\fi

\RequirePackage[TU]{fontenc}

\begingroup
  \normalsize
  \calculate@math@sizes
  \csname S@\f@size\endcsname
  \xdef\umath@sf{\strip@pt\dimexpr0.5\dimexpr\ssf@size pt+\sf@size pt\relax\relax}
  \xdef\umath@tf{\strip@pt\dimexpr0.5\dimexpr \sf@size pt+\tf@size pt\relax\relax}
\endgroup

\newcommand\mathencodingdefault{TU}
\newcommand\mathseriesdefault{m}
\newcommand\mathshapedefault{m}

\newcount\umath@scalecount
\umath@scalecount=0\relax

\newcommand\DeclareMathFamily[3]{%
  \def\umath@fontname{#1}%
  \def\@tmpa{#3}\def\@tmpb{}%
  \ifx\@tmpa\@tmpb
    \def\umath@family{#2}%
  \else
    \def\umath@family{#2-map-#3}%
  \fi
  \edef\umathfontdecl{[\umath@fontname]}%
  \edef\umath@mode;{\ifdefined\directlua mode=base;\fi}%
  \edef\umath@mapping;{\ifdefined\directlua\else mapping=mapping_math_#3\fi}%
  \DeclareFontFamily{\mathencodingdefault}{\umath@family}{}%
  \edef\umath@tmpa{\noexpand\DeclareFontShape
    {\mathencodingdefault}
    {\umath@family}
    {\mathseriesdefault}
    {\mathshapedefault}
    {<-\umath@sf>%
     s*[1.\umath@paddednum{\umath@scalecount+0}]%
     "\umathfontdecl:\umath@mode;script=math;+ssty=1;\umath@mapping;"%
     <\umath@sf-\umath@tf>%
     s*[1.\umath@paddednum{\umath@scalecount+1}]%
     "\umathfontdecl:\umath@mode;script=math;+ssty=0;\umath@mapping;"%
     <\umath@tf->%
     s*[1.\umath@paddednum{\umath@scalecount+2}]%
     "\umathfontdecl:\umath@mode;script=math;\umath@mapping;"}
    {}}\umath@tmpa
  \umath@scalecount=\numexpr\umath@scalecount+3\relax
  \ifdefined\directlua
    \expandafter\gdef\csname
        \mathencodingdefault/\umath@family/\mathseriesdefault/\mathshapedefault+preloadhook%
      \endcsname{\umath@fonthook{#3}}
  \fi
}

\def\umath@paddednum#1{%
  \ifnum\numexpr#1\relax<10 000\number\numexpr#1\relax\else
  \ifnum\numexpr#1\relax<100 00\number\numexpr#1\relax\else
  \ifnum\numexpr#1\relax<1000 0\number\numexpr#1\relax\fi\fi\fi
}

\def\DeclUSymbolFont#1#2#3#4{%
  \def\@tmpa{#4}\def\@tmpb{}%
  \ifx\@tmpa\@tmpb
    \def\umath@family{#3}%
  \else
    \def\umath@family{#3-map-#4}%
  \fi
  \ifcsname \mathencodingdefault+\umath@family\endcsname\else
    \DeclareMathFamily{#1}{#3}{#4}%
  \fi
  \DeclareSymbolFont
    {#2}
    {\mathencodingdefault}
    {\umath@family}
    {\mathseriesdefault}
    {\mathshapedefault}%
}
\def\DeclUMathAlph#1#2#3#4{%
  \def\@tmpa{#4}\def\@tmpb{}%
  \ifx\@tmpa\@tmpb
    \def\umath@family{#3}%
  \else
    \def\umath@family{#3-map-#4}%
  \fi
  \ifcsname \mathencodingdefault+\umath@family\endcsname\else
    \DeclareMathFamily{#1}{#3}{#4}%
  \fi
  \DeclareMathAlphabet
    {#2}
    {\mathencodingdefault}
    {\umath@family}
    {\mathseriesdefault}
    {\mathshapedefault}%
}


% need to fix all the NFSS math assignments

\def\Uset@mathchar#1#2#3#4{%
  \global\Umathcode`#2="\mathchar@type#3 #1 #4\relax}

\def\Uset@mathsymbol#1#2#3#4{%
  \global\Umathchardef#2 "\mathchar@type#3 #1 #4\relax}

\def\Uset@mathdelimiter#1#2#3#4{%
  \xdef#2{\Udelimiter "\mathchar@type#3 #1 #4\relax}%
}
\def\Uset@@mathdelimiter#1#2#3{%
  \global\Udelcode`#2=#1 #3\relax}


\def\uDeclareMathSymbol#1#2#3#4{%
  \expandafter\in@\csname sym#3\expandafter\endcsname
     \expandafter{\group@list}%
  \ifin@
    \begingroup
      \if\relax\noexpand#1% is command?
        \edef\reserved@a{\noexpand\in@{\string\mathchar}{\meaning#1}}%
        \reserved@a
        \ifin@
          \expandafter\Uset@mathsymbol
             \csname sym#3\endcsname{#1}{#2}{#4}%
          \@font@info{Redeclaring math symbol \string#1}%
        \else
            \expandafter\ifx
            \csname\expandafter\@gobble\string#1\endcsname
            \relax
            \expandafter\Uset@mathsymbol
               \csname sym#3\endcsname{#1}{#2}{#4}%
          \else
            \@latex@error{Command ‘\string#1’ already defined}\@eha
          \fi
\fi \else
        \expandafter\Uset@mathchar
          \csname sym#3\endcsname{#1}{#2}{#4}%
      \fi
    \endgroup
  \else
    \@latex@error{Symbol font ‘#3’ is not defined}\@eha
\fi
}




\def\uDeclareMathDelimiter#1{%
  \if\relax\noexpand#1%
    \expandafter\u@DeclareMathDelimiter
  \else
    \expandafter\u@xDeclareMathDelimiter
  \fi
  #1}

\def\u@DeclareMathDelimiter#1#2#3#4{%
  \expandafter\in@\csname sym#3\expandafter\endcsname
     \expandafter{\group@list}%
  \ifin@
      \begingroup
        \edef\reserved@a{\noexpand\in@{\string\delimiter}{\meaning#1}}%
        \reserved@a
        \ifin@
          \expandafter\Uset@mathdelimiter
             \csname sym#3\endcsname{#1}{#2}{#4}%
          \@font@info{Redeclaring math delimiter \string#1}%
        \else
            \expandafter\ifx
            \csname\expandafter\@gobble\string#1\endcsname
            \relax
            \expandafter\Uset@mathdelimiter
              \csname sym#3\endcsname{#1}{#2}{#4}%
          \else
            \@latex@error{Command `\string#1' already defined}\@eha
          \fi
        \fi
      \endgroup
  \else
    \@latex@error{Symbol font `#3' is not defined}\@eha
  \fi
}

\def\u@xDeclareMathDelimiter#1#2#3#4{%
  \expandafter\in@\csname sym#3\expandafter\endcsname
     \expandafter{\group@list}%
  \ifin@
      \begingroup
        \expandafter\Uset@@mathdelimiter
           \csname sym#3\endcsname{#1}{#4}%
      \endgroup
  \else
    \@latex@error{Symbol font `#3' is not defined}\@eha
  \fi
}



\def\uDeclareMathRadical#1#2#3{%
  \expandafter\ifx
       \csname\expandafter\@gobble\string#1\endcsname
       \relax
     \let#1\radical
  \fi
  \edef\reserved@a{\noexpand\in@{\string\radical}{\meaning#1}}%
  \reserved@a
  \ifin@
    \expandafter\in@\csname sym#2\expandafter\endcsname
       \expandafter{\group@list}%
    \ifin@
          \xdef#1{\Uradical \csname sym#2\endcsname\space#3\relax}%
    \else
      \@latex@error{Symbol font `#2' is not defined}\@eha
    \fi
  \else
    \@latex@error{Command `\string#1' already defined}\@eha
  \fi
}


\DeclareMathSymbol{\subset }{\mathrel }{symbols}{8834}
\DeclareMathSymbol{\in }{\mathrel }{symbols}{8712}
\DeclareMathSymbol{\supset }{\mathrel }{symbols}{8835}
\DeclareMathSymbol{\imath }{\mathord }{letters}{120484}
\DeclareMathSymbol{\biguplus }{\mathop }{largesymbols}{10756}
\DeclareMathSymbol{\wedge }{\mathbin }{symbols}{8743}
\DeclareMathSymbol{\forall }{\mathord }{symbols}{8704}
\DeclareMathSymbol{\bigcup }{\mathop }{largesymbols}{8899}
\DeclareMathSymbol{\prod }{\mathop }{largesymbols}{8719}
\DeclareMathSymbol{\ast }{\mathbin }{symbols}{8727}
\DeclareMathSymbol{\bigsqcup }{\mathop }{largesymbols}{10758}
\DeclareMathSymbol{\mp }{\mathbin }{symbols}{8723}
\DeclareMathSymbol{\diamondsuit }{\mathord }{symbols}{9826}
\DeclareMathSymbol{\parallel }{\mathrel }{symbols}{8741}
\DeclareMathDelimiter{\backslash }{\mathord }{symbols}{92}
\DeclareMathSymbol{\bigvee }{\mathop }{largesymbols}{8897}
\DeclareMathSymbol{\bigodot }{\mathop }{largesymbols}{10752}
\DeclareMathSymbol{\bot }{\mathord }{symbols}{8869}
\DeclareMathDelimiter{\updownarrow }{\mathrel }{symbols}{8597}
\DeclareMathSymbol{\geq }{\mathrel }{symbols}{8805}
\DeclareMathDelimiter{\lfloor }{\mathopen }{symbols}{8970}
\DeclareMathSymbol{\propto }{\mathrel }{symbols}{8733}
\DeclareMathSymbol{\ni }{\mathrel }{symbols}{8715}
\DeclareMathSymbol{\flat }{\mathord }{letters}{9837}
\DeclareMathSymbol{\top }{\mathord }{symbols}{8868}
\DeclareMathSymbol{\mid }{\mathrel }{symbols}{8739}
\DeclareMathSymbol{\infty }{\mathord }{symbols}{8734}
\DeclareMathSymbol{\nearrow }{\mathrel }{symbols}{8599}
\DeclareMathSymbol{\bigotimes }{\mathop }{largesymbols}{10754}
\DeclareMathDelimiter{\lgroup }{\mathopen }{largesymbols}{10222}
\DeclareMathSymbol{\wp }{\mathord }{letters}{8472}
\DeclareMathSymbol{\bigcap }{\mathop }{largesymbols}{8898}
\DeclareMathSymbol{\sqsubseteq }{\mathrel }{symbols}{8849}
\DeclareMathSymbol{\triangleright }{\mathbin }{letters}{9655}
\DeclareMathSymbol{\mathdollar }{\mathord }{operators}{36}
\DeclareMathSymbol{\ominus }{\mathbin }{symbols}{8854}
\DeclareMathSymbol{\aleph }{\mathord }{symbols}{8501}
\DeclareMathSymbol{\jmath }{\mathord }{letters}{120485}
\DeclareMathSymbol{\clubsuit }{\mathord }{symbols}{9827}
\DeclareMathSymbol{\cup }{\mathbin }{symbols}{8746}
\DeclareMathSymbol{\uplus }{\mathbin }{symbols}{8846}
\DeclareMathSymbol{\odot }{\mathbin }{symbols}{8857}
\DeclareMathSymbol{\leftrightarrow }{\mathrel }{symbols}{8596}
\DeclareMathSymbol{\sqsupseteq }{\mathrel }{symbols}{8850}
\DeclareMathDelimiter{\rfloor }{\mathclose }{symbols}{8971}
\DeclareMathSymbol{\le }{\mathrel }{symbols}{8804}
\DeclareMathSymbol{\perp }{\mathrel }{symbols}{10178}
\DeclareMathSymbol{\cap }{\mathbin }{symbols}{8745}
\DeclareMathDelimiter{\rceil }{\mathclose }{symbols}{8969}
\DeclareMathSymbol{\succeq }{\mathrel }{symbols}{10928}
\DeclareMathSymbol{\oslash }{\mathbin }{symbols}{8856}
\DeclareMathSymbol{\bigtriangleup }{\mathbin }{symbols}{9651}
\DeclareMathSymbol{\ell }{\mathord }{letters}{8467}
\DeclareMathSymbol{\sharp }{\mathord }{letters}{9839}
\DeclareMathSymbol{\sum }{\mathop }{largesymbols}{8721}
\DeclareMathSymbol{\sim }{\mathrel }{symbols}{8764}
\DeclareMathSymbol{\land }{\mathbin }{symbols}{8743}
\DeclareMathSymbol{\smile }{\mathrel }{letters}{8995}
\DeclareMathSymbol{\times }{\mathbin }{symbols}{215}
\DeclareMathSymbol{\rightharpoondown }{\mathrel }{letters}{8641}
\DeclareMathSymbol{\Rightarrow }{\mathrel }{symbols}{8658}
\DeclareMathSymbol{\natural }{\mathord }{letters}{9838}
\DeclareMathSymbol{\Leftrightarrow }{\mathrel }{symbols}{8660}
\DeclareMathDelimiter{\uparrow }{\mathrel }{symbols}{8593}
\DeclareMathDelimiter{\Downarrow }{\mathrel }{symbols}{8659}
\DeclareMathDelimiter{\downarrow }{\mathrel }{symbols}{8595}
\DeclareMathDelimiter{\rgroup }{\mathclose }{largesymbols}{10223}
\DeclareMathDelimiter{\lceil }{\mathopen }{symbols}{8968}
\DeclareMathDelimiter{\lbrace }{\mathopen }{symbols}{123}
\DeclareMathDelimiter{\rbrace }{\mathclose }{symbols}{125}
\DeclareMathDelimiter{\langle }{\mathopen }{symbols}{10216}
\DeclareMathSymbol{\vee }{\mathbin }{symbols}{8744}
\DeclareMathSymbol{\gg }{\mathrel }{symbols}{8811}
\DeclareMathDelimiter{\rangle }{\mathclose }{symbols}{10217}
\DeclareMathDelimiter{\Updownarrow }{\mathrel }{symbols}{8661}
\DeclareMathDelimiter{\Uparrow }{\mathrel }{symbols}{8657}
\DeclareMathSymbol{\triangleleft }{\mathbin }{letters}{9665}
\DeclareMathSymbol{\pm }{\mathbin }{symbols}{177}
\DeclareMathDelimiter{\|}{\mathord }{symbols}{8214}
\DeclareMathSymbol{\succ }{\mathrel }{symbols}{8827}
\DeclareMathSymbol{\ge }{\mathrel }{symbols}{8805}
\DeclareMathDelimiter{\rmoustache }{\mathclose }{largesymbols}{9137}
\DeclareMathSymbol{\frown }{\mathrel }{letters}{8994}
\DeclareMathSymbol{\neg }{\mathord }{symbols}{172}
\DeclareMathSymbol{\owns }{\mathrel }{symbols}{8715}
\DeclareMathSymbol{\simeq }{\mathrel }{symbols}{8771}
\DeclareMathSymbol{\cdotp }{\mathpunct }{symbols}{183}
\DeclareMathSymbol{\ll }{\mathrel }{symbols}{8810}
\DeclareMathSymbol{\div }{\mathbin }{symbols}{247}
\DeclareMathSymbol{\equiv }{\mathrel }{symbols}{8801}
\DeclareMathSymbol{\partial }{\mathord }{letters}{8706}
\DeclareMathSymbol{\oplus }{\mathbin }{symbols}{8853}
\DeclareMathSymbol{\sqcap }{\mathbin }{symbols}{8851}
\DeclareMathSymbol{\dashv }{\mathrel }{symbols}{8867}
\DeclareMathSymbol{\setminus }{\mathbin }{symbols}{10741}
\DeclareMathSymbol{\leftharpoondown }{\mathrel }{letters}{8637}
\DeclareMathSymbol{\leftharpoonup }{\mathrel }{letters}{8636}
\DeclareMathSymbol{\bigwedge }{\mathop }{largesymbols}{8896}
\DeclareMathSymbol{\heartsuit }{\mathord }{symbols}{9825}
\DeclareMathSymbol{\supseteq }{\mathrel }{symbols}{8839}
\DeclareMathDelimiter{\lmoustache }{\mathopen }{largesymbols}{9136}
\DeclareMathSymbol{\rightharpoonup }{\mathrel }{letters}{8640}
\DeclareMathSymbol{\asymp }{\mathrel }{symbols}{8781}
\DeclareMathSymbol{\searrow }{\mathrel }{symbols}{8600}
\DeclareMathSymbol{\swarrow }{\mathrel }{symbols}{8601}
\DeclareMathSymbol{\rightarrow }{\mathrel }{symbols}{8594}
\DeclareMathSymbol{\gets }{\mathrel }{symbols}{8592}
\DeclareMathSymbol{\leftarrow }{\mathrel }{symbols}{8592}
\DeclareMathDelimiter{\Vert }{\mathord }{symbols}{8214}
\DeclareMathSymbol{\wr }{\mathbin }{symbols}{8768}
\DeclareMathSymbol{\varbigtriangleup }{\mathbin }{symbols}{9651}
\DeclareMathSymbol{\subseteq }{\mathrel }{symbols}{8838}
\DeclareMathSymbol{\preceq }{\mathrel }{symbols}{10927}
\DeclareMathSymbol{\vdash }{\mathrel }{symbols}{8866}
\DeclareMathSymbol{\prec }{\mathrel }{symbols}{8826}
\DeclareMathSymbol{\varbigtriangledown }{\mathbin }{symbols}{9661}
\DeclareMathSymbol{\cdot }{\mathbin }{symbols}{8901}
\DeclareMathSymbol{\leq }{\mathrel }{symbols}{8804}
\DeclareMathSymbol{\not }{\mathrel }{symbols}{824}
\DeclareMathSymbol{\exists }{\mathord }{symbols}{8707}
\DeclareMathSymbol{\Leftarrow }{\mathrel }{symbols}{8656}
\DeclareMathSymbol{\nwarrow }{\mathrel }{symbols}{8598}
\DeclareMathSymbol{\spadesuit }{\mathord }{symbols}{9824}
\DeclareMathSymbol{\bigoplus }{\mathop }{largesymbols}{10753}
\DeclareMathSymbol{\to }{\mathrel }{symbols}{8594}
\DeclareMathSymbol{\star }{\mathbin }{letters}{8902}
\DeclareMathSymbol{\otimes }{\mathbin }{symbols}{8855}
\DeclareMathSymbol{\amalg }{\mathbin }{symbols}{10815}
\DeclareMathSymbol{\sqcup }{\mathbin }{symbols}{8852}
\DeclareMathSymbol{\ddagger }{\mathbin }{symbols}{8225}
\DeclareMathSymbol{\lnot }{\mathord }{symbols}{172}
\DeclareMathSymbol{\bigtriangledown }{\mathbin }{symbols}{9661}
\DeclareMathDelimiter{\vert }{\mathord }{symbols}{124}
\DeclareMathSymbol{\lor }{\mathbin }{symbols}{8744}
\DeclareMathSymbol{\prime }{\mathord }{symbols}{8242}
\DeclareMathSymbol{\coprod }{\mathop }{largesymbols}{8720}
\DeclareMathSymbol{\Im }{\mathord }{symbols}{8465}
\DeclareMathSymbol{\Re }{\mathord }{symbols}{8476}
\DeclareMathSymbol{\approx }{\mathrel }{symbols}{8776}
\DeclareMathSymbol{\nabla }{\mathord }{symbols}{8711}
\DeclareMathSymbol{\dagger }{\mathbin }{symbols}{8224}

\endinput

\DeclareMathSymbol{\kappa }{...}{...}{...}
\DeclareMathSymbol{\rho }{...}{...}{...}
\DeclareMathSymbol{\mu }{...}{...}{...}
\DeclareMathSymbol{\alpha }{...}{...}{...}
\DeclareMathSymbol{\bracelu }{...}{...}{...}
\DeclareMathSymbol{\nu }{...}{...}{...}
\DeclareMathSymbol{!}{...}{...}{...}
\DeclareMathSymbol{\Lambda }{...}{...}{...}
\DeclareMathSymbol{\omega }{...}{...}{...}
\DeclareMathSymbol{3}{...}{...}{...}
\DeclareMathSymbol{1}{...}{...}{...}
\DeclareMathSymbol{2}{...}{...}{...}
\DeclareMathSymbol{/}{...}{...}{...}
\DeclareMathSymbol{-}{...}{...}{...}
\DeclareMathSymbol{.}{...}{...}{...}
\DeclareMathSymbol{,}{...}{...}{...}
\DeclareMathDelimiter{)}{...}{...}{...}
\DeclareMathSymbol{*}{...}{...}{...}
\DeclareMathDelimiter{(}{...}{...}{...}
\DeclareMathSymbol{\lambda }{...}{...}{...}
\DeclareMathSymbol{\Omega }{...}{...}{...}
\DeclareMathRadical{\sqrtsign }{...}{...}{...}
\DeclareMathSymbol{c}{...}{...}{...}
\DeclareMathSymbol{d}{...}{...}{...}
\DeclareMathSymbol{a}{...}{...}{...}
\DeclareMathSymbol{b}{...}{...}{...}
\DeclareMathDelimiter{]}{...}{...}{...}
\DeclareMathDelimiter{[}{...}{...}{...}
\DeclareMathDelimiter{\}{...}{...}{...}
\DeclareMathSymbol{Y}{...}{...}{...}
\DeclareMathSymbol{Z}{...}{...}{...}
\DeclareMathSymbol{W}{...}{...}{...}
\DeclareMathSymbol{X}{...}{...}{...}
\DeclareMathSymbol{U}{...}{...}{...}
\DeclareMathSymbol{V}{...}{...}{...}
\DeclareMathSymbol{s}{...}{...}{...}
\DeclareMathSymbol{t}{...}{...}{...}
\DeclareMathSymbol{q}{...}{...}{...}
\DeclareMathSymbol{r}{...}{...}{...}
\DeclareMathSymbol{o}{...}{...}{...}
\DeclareMathSymbol{p}{...}{...}{...}
\DeclareMathSymbol{m}{...}{...}{...}
\DeclareMathSymbol{n}{...}{...}{...}
\DeclareMathSymbol{k}{...}{...}{...}
\DeclareMathSymbol{l}{...}{...}{...}
\DeclareMathSymbol{i}{...}{...}{...}
\DeclareMathSymbol{j}{...}{...}{...}
\DeclareMathSymbol{g}{...}{...}{...}
\DeclareMathSymbol{h}{...}{...}{...}
\DeclareMathSymbol{e}{...}{...}{...}
\DeclareMathSymbol{f}{...}{...}{...}
\DeclareMathSymbol{C}{...}{...}{...}
\DeclareMathSymbol{A}{...}{...}{...}
\DeclareMathSymbol{B}{...}{...}{...}
\DeclareMathSymbol{?}{...}{...}{...}
\DeclareMathSymbol{=}{...}{...}{...}
\DeclareMathSymbol{>}{...}{...}{...}
\DeclareMathSymbol{;}{...}{...}{...}
\DeclareMathSymbol{<}{...}{...}{...}
\DeclareMathSymbol{9}{...}{...}{...}
\DeclareMathSymbol{:}{...}{...}{...}
\DeclareMathSymbol{7}{...}{...}{...}
\DeclareMathSymbol{8}{...}{...}{...}
\DeclareMathSymbol{5}{...}{...}{...}
\DeclareMathSymbol{6}{...}{...}{...}
\DeclareMathSymbol{S}{...}{...}{...}
\DeclareMathSymbol{T}{...}{...}{...}
\DeclareMathSymbol{Q}{...}{...}{...}
\DeclareMathSymbol{R}{...}{...}{...}
\DeclareMathSymbol{O}{...}{...}{...}
\DeclareMathSymbol{P}{...}{...}{...}
\DeclareMathSymbol{M}{...}{...}{...}
\DeclareMathSymbol{N}{...}{...}{...}
\DeclareMathSymbol{K}{...}{...}{...}
\DeclareMathSymbol{I}{...}{...}{...}
\DeclareMathSymbol{J}{...}{...}{...}
\DeclareMathSymbol{G}{...}{...}{...}
\DeclareMathSymbol{H}{...}{...}{...}
\DeclareMathSymbol{E}{...}{...}{...}
\DeclareMathSymbol{F}{...}{...}{...}
\DeclareMathSymbol{\Phi }{...}{...}{...}
\DeclareMathSymbol{\Xi }{...}{...}{...}
\DeclareMathSymbol{\tau }{...}{...}{...}
\DeclareMathDelimiter{|}{...}{...}{...}
\DeclareMathSymbol{y}{...}{...}{...}
\DeclareMathSymbol{z}{...}{...}{...}
\DeclareMathSymbol{w}{...}{...}{...}
\DeclareMathSymbol{x}{...}{...}{...}
\DeclareMathSymbol{u}{...}{...}{...}
\DeclareMathSymbol{v}{...}{...}{...}
\DeclareMathSymbol{\phi }{...}{...}{...}
\DeclareMathSymbol{\circ }{...}{...}{...}
\DeclareMathSymbol{\bullet }{...}{...}{...}
\DeclareMathSymbol{\varsigma }{...}{...}{...}
\DeclareMathSymbol{\colon }{...}{...}{...}
\DeclareMathSymbol{\emptyset }{...}{...}{...}
\DeclareMathSymbol{\eta }{...}{...}{...}
\DeclareMathSymbol{\mapstochar }{...}{...}{...}
\DeclareMathSymbol{\chi }{...}{...}{...}
\DeclareMathSymbol{\smallint }{...}{...}{...}
\DeclareMathSymbol{\braceru }{...}{...}{...}
\DeclareMathSymbol{\delta }{...}{...}{...}
\DeclareMathSymbol{\Theta }{...}{...}{...}
\DeclareMathSymbol{\braceld }{...}{...}{...}
\DeclareMathSymbol{\mathsection }{...}{...}{...}
\DeclareMathSymbol{\mathparagraph }{...}{...}{...}
\DeclareMathDelimiter{\bracevert }{...}{...}{...}
\DeclareMathSymbol{\ointop }{...}{...}{...}
\DeclareMathSymbol{\theta }{...}{...}{...}
\DeclareMathDelimiter{\Arrowvert }{...}{...}{...}
\DeclareMathDelimiter{\arrowvert }{...}{...}{...}
\DeclareMathSymbol{\bracerd }{...}{...}{...}
\DeclareMathSymbol{\iota }{...}{...}{...}
\DeclareMathSymbol{\varepsilon }{...}{...}{...}
\DeclareMathSymbol{\Delta }{...}{...}{...}
\DeclareMathSymbol{\ldotp }{...}{...}{...}
\DeclareMathSymbol{\lhook }{...}{...}{...}
\DeclareMathSymbol{\Gamma }{...}{...}{...}
\DeclareMathSymbol{\Pi }{...}{...}{...}
\DeclareMathSymbol{\beta }{...}{...}{...}
\DeclareMathSymbol{\rhook }{...}{...}{...}
\DeclareMathSymbol{\intop }{...}{...}{...}
\DeclareMathSymbol{\epsilon }{...}{...}{...}
\DeclareMathSymbol{\varpi }{...}{...}{...}
\DeclareMathSymbol{\varphi }{...}{...}{...}
\DeclareMathSymbol{\gamma }{...}{...}{...}
\DeclareMathSymbol{\Psi }{...}{...}{...}
\DeclareMathSymbol{\triangle }{...}{...}{...}
\DeclareMathSymbol{\Sigma }{...}{...}{...}
\DeclareMathSymbol{4}{...}{...}{...}
\DeclareMathSymbol{0}{...}{...}{...}
\DeclareMathSymbol{\upsilon }{...}{...}{...}
\DeclareMathSymbol{+}{...}{...}{...}
\DeclareMathSymbol{\sigma }{...}{...}{...}
\DeclareMathSymbol{\diamond }{...}{...}{...}
\DeclareMathSymbol{\bigcirc }{...}{...}{...}
\DeclareMathSymbol{D}{...}{...}{...}
\DeclareMathSymbol{\zeta }{...}{...}{...}
\DeclareMathSymbol{\psi }{...}{...}{...}
\DeclareMathSymbol{\xi }{...}{...}{...}
\DeclareMathSymbol{\pi }{...}{...}{...}
\DeclareMathSymbol{L}{...}{...}{...}
\DeclareMathSymbol{\Upsilon }{...}{...}{...}
\DeclareMathSymbol{\varrho }{...}{...}{...}
\DeclareMathSymbol{\vartheta }{...}{...}{...}

\input{umath2e-syms.tex}

\endinput
%    \end{macrocode}
%
%    \begin{macrocode}
%</tex>
%    \end{macrocode}
%
% \subsection{Lua}
%
% \begingroup
%
%  \begingroup\lccode`~=`_
%  \lowercase{\endgroup\let~}_
%  \catcode`_=12
%
%    \begin{macrocode}
%<*callback>
%    \end{macrocode}
%
%    \begin{macrocode}
-- require('serialise')
require('math-map-gen.lua')
require('umath-alphabet-mappings.lua')

orig_define_font=luatexbase.remove_from_callback('define_font','luaotfload.define_font')

function x_define_font(name,size,id)
  local thisfont=orig_define_font(name,size,id)
  this_mapping = math_maps[mathmapstr]
  if (this_mapping and type(thisfont)=='table') then
--[[ # this is what's in the font:
      io.write('\n\n')
      for k,v in pairs(thisfont) do
        if type(v) == 'table' then
          local count = 0
          for _ in pairs(v) do count = count + 1 end
          io.write(k,' (( table, length ',count,'))\n')
        else
          print(k,' = "',v,'"',' (',type(v),')')
        end
      end io.write('\n\n')  ERROR()
--]]
    for k,v in pairs(this_mapping) do
	    thisfont.characters[k]=subst_glyph(thisfont,v)
	  end
  end
  return thisfont
end

function subst_glyph(font,slot)
  if not( font.characters[slot] ) then  return nil  end

  local char={}
  local entries = {"width","top_accent","italic","height","depth"}

  -- if slot > 0x10000 then  serialise(font.characters[slot]) ERROR()  end
  for i,v in pairs(entries) do
    char[v] = font.characters[slot][v]
  end
  char.commands   = {{'char',slot}}
  return char
end

luatexbase.add_to_callback('define_font',x_define_font,'my_define_font')
%    \end{macrocode}
%
%    \begin{macrocode}
%</callback>
%    \end{macrocode}
%
%
%    \begin{macrocode}
%<*mappings>
%    \end{macrocode}
%
%    \begin{macrocode}

--[[
Start off with a definition for unicode glyph slot names.
These need to be consistent because they're used programmatically.
Note that we could have defined this in a more "nested"-like structure,
but this would have made it less clear that we're talking about input
and output characters here, not abstractions.
--]]

dofile('umath-glyph-names.lua')

latin_upper = {'A','B','C','D','E','F','G','H','I','J','K','L','M','N','O','P','Q','R','S','T','U','V','W','X','Y','Z'}
latin_lower = {'a','b','c','d','e','f','g','h','i','j','k','l','m','n','o','p','q','r','s','t','u','v','w','x','y','z'}

greek_upper = {'Alpha', 'Beta', 'Gamma', 'Delta', 'Epsilon', 'Zeta', 'Eta', 'Theta',
  'Iota', 'Kappa', 'Lambda', 'Mu', 'Nu', 'Xi', 'Omicron', 'Pi', 'Rho', 'Sigma', 'Tau',
  'Upsilon', 'Phi', 'Chi', 'Psi', 'Omega'}
greek_lower = {'alpha', 'beta', 'gamma', 'delta', 'epsilon', 'zeta', 'eta', 'theta',
  'iota', 'kappa', 'lambda', 'mu', 'nu', 'xi', 'omicron', 'pi', 'rho', 'sigma', 'tau',
  'upsilon', 'phi', 'chi', 'psi', 'omega'}

greek_sym = {'varTheta','nabla','partial','varsigma','vartheta','varphi','varpi','varkappa','varrho','varepsilon'}

num = {'zero','one','two','three','four','five','six','seven','eight','nine'}

--[[

# Define the mappings

Basically what we want here is to end up with a nested table structure that looks like

math_map = {
  ["italic"] = { ["mrmA"] = "mitA" , ["mrmB"] = "mrmB" , ... }
  ["bold"]   = { ["mrmA"] = "mbfA" , ["mrmB"] = "mbfB" , ... }
}

So to do this we define:

 * What the "name" of the math alphabet is ("italic", "bold", ...) (this is loop 1)
 * What the "prefix" of it is ("mit", "mbf")
 * What alphabets need to be included (this is loop 2)

Each alphabet is defined in terms of a series of slot names (this is loop 3).

This setup makes things nice and general, and we can define arbitrary alphabets as long as the
glyph naming scheme is consistent. If the glyph names are NOT consistent, we need to augment
the resulting structure manually. This is probably easier than making the construction code
any more convoluted!

--]]

mathalph = {}
mathalph.italic         = { ["prefix"] = "mit" ,       ["alphlist"] = {latin_upper,latin_lower,greek_upper,greek_lower,greek_sym} }
mathalph.bold           = { ["prefix"] = "mbf" ,       ["alphlist"] = {latin_upper,latin_lower,greek_upper,greek_lower,greek_sym,num} }
mathalph.bolditalic     = { ["prefix"] = "mbfit" ,     ["alphlist"] = {latin_upper,latin_lower,greek_upper,greek_lower,greek_sym} }
mathalph.scr            = { ["prefix"] = "mscr" ,      ["alphlist"] = {latin_upper,latin_lower} }
mathalph.boldscr        = { ["prefix"] = "mbfscr" ,    ["alphlist"] = {latin_upper,latin_lower} }
mathalph.frak           = { ["prefix"] = "mfrak" ,     ["alphlist"] = {latin_upper,latin_lower} }
mathalph.boldfrak       = { ["prefix"] = "mbffrak" ,   ["alphlist"] = {latin_upper,latin_lower} }
mathalph.bb             = { ["prefix"] = "Bbb" ,       ["alphlist"] = {latin_upper,latin_lower,num} }
mathalph.tt             = { ["prefix"] = "mtt" ,       ["alphlist"] = {latin_upper,latin_lower,num} }
mathalph.sans           = { ["prefix"] = "msans" ,     ["alphlist"] = {latin_upper,latin_lower,num} }
mathalph.sansitalic     = { ["prefix"] = "mitsans" ,   ["alphlist"] = {latin_upper,latin_lower} }
mathalph.boldsans       = { ["prefix"] = "mbfsans" ,   ["alphlist"] = {latin_upper,latin_lower,greek_upper,greek_lower,greek_sym,num} }
mathalph.boldsansitalic = { ["prefix"] = "mbfitsans" , ["alphlist"] = {latin_upper,latin_lower,greek_upper,greek_lower,greek_sym} }

math_map = {}
input_prefix = 'mrm'

for mathalph_name,thismathalph in pairs(mathalph) do
  math_map[mathalph_name] = {}
  for i,alphlist in pairs(thismathalph.alphlist) do
    for ii,slotname in pairs(alphlist) do
      math_map[mathalph_name][input_prefix..slotname] = thismathalph.prefix..slotname
    end
  end
end


--[[

# Write the tables to file in appropriate ways

Three files needed:

1. The Lua table mapping numeric glyph slots to glyph slots for each math alphabet.
   Basically a slimmed down `serialise`.

2. The XeTeX font mapping files.

3. For convenience, a shell script for compiling the XeTeX font mapping files so XeTeX can read them.

--]]

local f = assert(io.open("umath-alphabet-mappings.lua", "w"))
f:write("-- this file is automatically generated; do not edit","\n\n")
f:write("math_maps = {","\n")
for kk,vv in pairs(math_map) do
  f:write("  [\"",kk,"\"] = {","\n")
  for k,v in pairs(vv) do
    slot = math_sym_names[k] or k
    val  = math_sym_names[v] or v
    f:write("    [0x",slot,"] = 0x",val," ,","\n")
  end
  f:write("  },","\n")
end
f:write('}','\n')
f:close()

teckit_prefix = "mapping_math_"
for kk,vv in pairs(math_map) do
  tecmapname = teckit_prefix..kk
  local f = assert(io.open(tecmapname..".map", "w"))
  f:write("LHSName \"ascii\"","\n","RHSName \"","unicode-maths-",kk,"\"","\n","pass(Unicode)","\n\n")
  for k,v in pairs(vv) do
    slot = math_sym_names[k] or k
    val  = math_sym_names[v] or v
    f:write("U+",slot," <> U+",val," ;\n")
  end
  f:close()
  os.execute("teckit_compile " .. tecmapname .. "\n")
end
%    \end{macrocode}
%
%    \begin{macrocode}
%</mappings>
%    \end{macrocode}
%
% \engroup
%
% \Finale
